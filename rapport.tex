\documentclass[a4paper, 12pt]{report}

% --- PACKAGES ---
\usepackage[utf8]{inputenc}
\usepackage[T1]{fontenc}
\usepackage[french]{babel}
\usepackage{geometry}
\usepackage{graphicx}
\usepackage{booktabs}
\usepackage{hyperref}
\usepackage{float}
\usepackage{caption}
\usepackage{subcaption}
\usepackage{xcolor}
\usepackage{titlesec}

% --- CONFIGURATION ---
\geometry{hmargin=2.5cm, vmargin=2.5cm}
\setlength{\parskip}{1em}
\setcounter{tocdepth}{2}

% Commande pour les instructions aux étudiants
\newcommand{\instruction}[2]{\textbf{\textcolor{blue}{[#1 : #2]}}}
\newcommand{\todo}[1]{\textit{\textcolor{red}{(TODO: #1)}}}

% --- METADONNÉES ---
\title{\textbf{Analyse de Données Avancée}\\
\Large Airbnb à New York : Économie de partage ou para-hôtellerie ?}
\author{
    \textbf{Groupe de Projet M1} \\[0.5cm]
    Marc-Antony TCHICOU \\
    Kelvin WONG \\
    Mouhamed CISSE \\
    Alexis SCHNEIDER \\
    Mohammed-Yahya DAMI
}
\date{ENSIIE -- Année 2024-2025}

\begin{document}

% --- PAGE DE TITRE ---
\maketitle

% --- TABLE DES MATIÈRES ---
\tableofcontents
\newpage

% --- INTRODUCTION ---
\chapter*{Introduction Générale}
\addcontentsline{toc}{chapter}{Introduction Générale}
\textbf{Problématique :} Dans quelle mesure le marché Airbnb new-yorkais reflète-t-il encore une économie de partage, et comment l'activité des hôtes se structure-t-elle géographiquement et économiquement ?

Pour répondre à cette question, nous avons divisé l'analyse en cinq axes majeurs, répartis entre les membres du groupe.

\newpage

% ====================================================================
% PARTIE 1 : PRÉPARATION ET ANALYSE DESCRIPTIVE
% Étudiant responsable : [Nom Étudiant 1]
% ====================================================================
\chapter{Panorama du Marché : Nettoyage et Statistiques}
\textit{Responsable : Étudiant 1}

\section{Préparation des Données}
\instruction{Étudiant 1}{Décrire le nettoyage (valeurs manquantes, dates) et les transformations.}

Le dataset initial comportait du bruit qu'il a fallu traiter :
\begin{itemize}
    \item \textbf{Nettoyage des prix :} Suppression des valeurs aberrantes (prix = 0\$) et transformation logarithmique (\texttt{log\_price}) pour réduire l'asymétrie.
    \item \textbf{Segmentation des hôtes :} Création de la variable \texttt{host\_type} :
    \begin{itemize}
        \item \textit{Occasionnel} (1 annonce), \textit{Multi-propriétaire} (2-5), \textit{Professionnel} (>5).
    \end{itemize}
\end{itemize}

\section{Analyse Descriptive}
\instruction{Étudiant 1}{Analyser la distribution des prix et les parts de marché.}

L'analyse univariée et bivariée met en évidence plusieurs faits stylisés :
\begin{enumerate}
    \item \textbf{La fracture des prix :} Manhattan est nettement plus cher que les autres arrondissements.
    \item \textbf{La domination du logement entier :} Ce type de bien représente la majorité de l'offre et des revenus, s'éloignant du concept de "chambre chez l'habitant".
\end{enumerate}

\section{Tests Statistiques}
\instruction{Étudiant 1}{Insérer les résultats du test de Kruskal-Wallis.}

Le test de Kruskal-Wallis confirme que les différences de prix et de disponibilité entre les types d'hôtes sont statistiquement significatives (p-value $\approx 0$).

\newpage

% ====================================================================
% PARTIE 2 : ANALYSE FACTORIELLE (ACP)
% Étudiant responsable : [Nom Étudiant 2]
% ====================================================================
\chapter{Analyse en Composantes Principales (ACP)}
\textit{Responsable : Étudiant 2}

\section{Choix des Variables Actives}
\instruction{Étudiant 2}{Lister les variables utilisées pour l'ACP.}

Nous avons retenu les variables quantitatives suivantes décrivant l'activité : \texttt{price}, \texttt{minimum\_nights}, \texttt{availability\_365}, \texttt{number\_of\_reviews}, \texttt{calculated\_host\_listings\_count}.

\section{Interprétation des Axes}
L'analyse du cercle des corrélations révèle deux dimensions latentes :
\begin{itemize}
    \item \textbf{Axe 1 (28.20\% variance) - L'intensité commerciale :} Oppose les logements disponibles toute l'année (type hôtel) aux logements peu disponibles.
    \item \textbf{Axe 2 (23.34\% variance) - La barrière à l'entrée :} Oppose les logements chers à forte contrainte (nuits min) aux logements populaires très commentés.
\end{itemize}
\todo{Insérer le graphique du cercle des corrélations et la projection des individus.}

\newpage

% ====================================================================
% PARTIE 3 : ANALYSE CANONIQUE (CCA)
% Étudiant responsable : [Nom Étudiant 3]
% ====================================================================
\chapter{Analyse Canonique des Corrélations (CCA)}
\textit{Responsable : Étudiant 3}

\section{Objectif et Définition des Groupes}
\instruction{Étudiant 3}{Expliquer le but de la CCA : lier deux sets de variables.}

L'Analyse Canonique des Corrélations (CCA) vise à étudier les relations entre deux groupes de variables distincts pour voir s'ils partagent une information commune.
\begin{itemize}
    \item \textbf{Groupe X (Caractéristiques de l'Hôte/Offre) :} \todo{Ex: calculated\_host\_listings\_count, availability\_365...}
    \item \textbf{Groupe Y (Performance/Usage) :} \todo{Ex: number\_of\_reviews, reviews\_per\_month, price...}
\end{itemize}

\section{Résultats et Corrélations Canoniques}
\instruction{Étudiant 3}{Interpréter les coefficients canoniques.}

Nous avons identifié les paires de variables canoniques maximisant la corrélation.
\todo{Insérer le tableau des corrélations ou le graphique de la CCA.}

\section{Interprétation Économique}
Les résultats suggèrent que la structure de l'offre (disponibilité, professionnalisme de l'hôte) est fortement corrélée à la performance de l'annonce (fréquence des commentaires), validant l'hypothèse d'une industrialisation efficace.

\newpage

% ====================================================================
% PARTIE 4 : CLUSTERING (SEGMENTATION)
% Étudiant responsable : [Nom Étudiant 4]
% ====================================================================
\chapter{Segmentation des Hôtes (Clustering)}
\textit{Responsable : Étudiant 4}

\section{Classification K-Means}
Sur la base des composantes principales, nous avons identifié 4 clusters distincts d'annonces.

\section{Profils Types}
\instruction{Étudiant 4}{Décrire les clusters (voir slide 15/16).}
\begin{itemize}
    \item \textbf{Cluster 0 :} Offre standard moyenne gamme.
    \item \textbf{Cluster 1 (Business) :} Disponibilité maximale, géré par des multi-propriétaires.
    \item \textbf{Cluster 2 (Partage) :} Faible disponibilité, prix bas, hôtes occasionnels.
    \item \textbf{Cluster 3 :} Offre luxe / long séjour.
\end{itemize}

\section{Analyse Croisée}
\todo{Insérer les graphiques "stacked bar" montrant la répartition des types d'appartements et des arrondissements par cluster.}

\newpage

% ====================================================================
% PARTIE 5 : ANALYSE SPATIALE ET MODÉLISATION
% Étudiant responsable : [Nom Étudiant 5]
% ====================================================================
\chapter{Analyse Spatiale et Modélisation}
\textit{Responsable : Étudiant 5}

\section{DBSCAN : La Réalité Géographique}
\instruction{Étudiant 5}{Commenter le clustering par densité.}

L'algorithme DBSCAN révèle un "super-cluster" touristique ignorant les frontières administratives entre Manhattan et Brooklyn, concentrant l'activité professionnelle.

\section{Régression des Prix (OLS)}
\instruction{Étudiant 5}{Analyser les déterminants du prix.}

Le modèle économétrique ($R^2 \approx 0.50$) montre que :
\begin{itemize}
    \item La localisation à Manhattan est le facteur n°1 (+55\% sur le prix).
    \item La privatisation du logement (Entire home) est le facteur n°2.
    \item Les professionnels ne sont pas intrinsèquement plus chers à bien égal, mais génèrent plus de volume.
\end{itemize}

\chapter*{Conclusion Générale}
L'étude confirme la dualité du marché new-yorkais entre une périphérie de partage et un centre hyper-professionnalisé.

\end{document}
