\documentclass[a4paper, 12pt]{report}

% --- PACKAGES ---
\usepackage[utf8]{inputenc}
\usepackage[T1]{fontenc}
\usepackage[french]{babel}
\usepackage{geometry}
\usepackage{graphicx}
\usepackage{booktabs}
\usepackage{hyperref}
\usepackage{float}
\usepackage{caption}
\usepackage{subcaption}
\usepackage{xcolor}
\usepackage{titlesec}

% --- CONFIGURATION ---
\geometry{hmargin=2.5cm, vmargin=2.5cm}
\setlength{\parskip}{1em}
\setcounter{tocdepth}{2}

% Commande pour les instructions aux étudiants
\newcommand{\instruction}[2]{\textbf{\textcolor{blue}{[#1 : #2]}}}
\newcommand{\todo}[1]{\textit{\textcolor{red}{(TODO: #1)}}}

% --- METADONNÉES ---
\title{\textbf{Analyse de Données Avancée}\\
\Large Airbnb à New York : Économie de partage ou para-hôtellerie ?}
\author{
    \textbf{Groupe de Projet M1} \\[0.5cm]
    Marc-Antony TCHICOU \\
    Kelvin WONG \\
    Mouhamed CISSE \\
    Alexis SCHNEIDER \\
    Mohammed-Yahya DAMI
}
\date{ENSIIE -- Année 2024-2025}

\begin{document}

% --- PAGE DE TITRE ---
\maketitle

% --- TABLE DES MATIÈRES ---
\tableofcontents
\newpage

% --- INTRODUCTION ---
\chapter*{Introduction Générale}
\addcontentsline{toc}{chapter}{Introduction Générale}
\textbf{Problématique :} Dans quelle mesure le marché Airbnb new-yorkais reflète-t-il encore une économie de partage, et comment l'activité des hôtes se structure-t-elle géographiquement et économiquement ?

Pour répondre à cette question, nous avons divisé l'analyse en cinq axes majeurs, répartis entre les membres du groupe.

\newpage

% ====================================================================
% PARTIE 1 : PRÉPARATION ET ANALYSE DESCRIPTIVE
% Étudiant responsable : [Nom Étudiant 1]
% ====================================================================
\chapter{Panorama du Marché : Nettoyage et Statistiques}
\textit{Responsable : Étudiant 1}

\section{Préparation des Données}
\instruction{Étudiant 1}{Décrire le nettoyage (valeurs manquantes, dates) et les transformations.}

Le dataset initial comportait du bruit qu'il a fallu traiter :
\begin{itemize}
    \item \textbf{Nettoyage des prix :} Suppression des valeurs aberrantes (prix = 0\$) et transformation logarithmique (\texttt{log\_price}) pour réduire l'asymétrie.
    \item \textbf{Segmentation des hôtes :} Création de la variable \texttt{host\_type} :
    \begin{itemize}
        \item \textit{Occasionnel} (1 annonce), \textit{Multi-propriétaire} (2-5), \textit{Professionnel} (>5).
    \end{itemize}
\end{itemize}

\section{Analyse Descriptive}
\instruction{Étudiant 1}{Analyser la distribution des prix et les parts de marché.}

L'analyse univariée et bivariée met en évidence plusieurs faits stylisés :
\begin{enumerate}
    \item \textbf{La fracture des prix :} Manhattan est nettement plus cher que les autres arrondissements.
    \item \textbf{La domination du logement entier :} Ce type de bien représente la majorité de l'offre et des revenus, s'éloignant du concept de "chambre chez l'habitant".
\end{enumerate}

\section{Tests Statistiques}
\instruction{Étudiant 1}{Insérer les résultats du test de Kruskal-Wallis.}

Le test de Kruskal-Wallis confirme que les différences de prix et de disponibilité entre les types d'hôtes sont statistiquement significatives (p-value $\approx 0$).

\newpage

% ====================================================================
% PARTIE 2 : ANALYSE FACTORIELLE (ACP)
% Étudiant responsable : [Nom Étudiant 2]
% ====================================================================
\chapter{Analyse en Composantes Principales (ACP)}
\textit{Responsable : Étudiant 2}

\section{Choix des Variables Actives}
\instruction{Étudiant 2}{Lister les variables utilisées pour l'ACP.}

Nous avons retenu les variables quantitatives suivantes décrivant l'activité : \texttt{price}, \texttt{minimum\_nights}, \texttt{availability\_365}, \texttt{number\_of\_reviews}, \texttt{calculated\_host\_listings\_count}.

\section{Interprétation des Axes}
L'analyse du cercle des corrélations révèle deux dimensions latentes :
\begin{itemize}
    \item \textbf{Axe 1 (28.20\% variance) - L'intensité commerciale :} Oppose les logements disponibles toute l'année (type hôtel) aux logements peu disponibles.
    \item \textbf{Axe 2 (23.34\% variance) - La barrière à l'entrée :} Oppose les logements chers à forte contrainte (nuits min) aux logements populaires très commentés.
\end{itemize}
\todo{Insérer le graphique du cercle des corrélations et la projection des individus.}

\newpage

% ====================================================================
% PARTIE 3 : ANALYSE CANONIQUE (CCA)
% Étudiant responsable : [Nom Étudiant 3]
% ====================================================================
\chapter{Analyse Canonique des Corrélations (CCA)}
\textit{Responsable : Dami Mohammed-Yahya}

\section{Objectif et Définition des Groupes}

L'objectif de cette Analyse Canonique des Corrélations (CCA) est de déterminer s'il existe une structure de dépendance entre la stratégie adoptée par les hôtes et la réponse du marché. Contrairement à une corrélation simple, la CCA nous permet de construire des "variables latentes" (combinaisons linéaires) pour maximiser le lien entre deux groupes multidimensionnels.

Nous avons défini les deux groupes de variables suivants, standardisés au préalable :

\begin{itemize}
    \item \textbf{Groupe X (Stratégie de l'Hôte) :} Ce groupe représente les leviers d'action contrôlés par l'hôte.
    \begin{itemize}
        \item \texttt{availability\_365} : La disponibilité annuelle du logement.
        \item \texttt{calculated\_host\_listings\_count} : Le nombre d'annonces gérées (taille du portefeuille).
        \item \texttt{minimum\_nights} : La contrainte de durée de séjour.
    \end{itemize}
    
    \item \textbf{Groupe Y (Performance du Marché) :} Ce groupe représente la réponse économique et la demande.
    \begin{itemize}
        \item \texttt{number\_of\_reviews} et \texttt{reviews\_per\_month} : Indicateurs de volume et d'intensité d'usage.
        \item \texttt{log\_price} : Indicateur de valorisation financière (rente).
    \end{itemize}
\end{itemize}

\section{Résultats et Corrélations Canoniques}

L'analyse a permis d'extraire deux paires de variables canoniques significatives, révélant deux mécanismes économiques distincts et indépendants (orthogonaux).

\begin{table}[h!]
    \centering
    \begin{tabular}{|c|c|c|}
        \hline
        \textbf{Axe Canonique} & \textbf{Corrélation ($\lambda$)} & \textbf{Interprétation de la Force} \\
        \hline
        Variable Canonique 1 (CV1) & \textbf{0.326} & Lien structurel fort (Volume) \\
        Variable Canonique 2 (CV2) & \textbf{0.112} & Lien secondaire existant (Rente) \\
        \hline
    \end{tabular}
    \caption{Corrélations Canoniques entre les groupes X et Y}
    \label{tab:cca_correlations}
\end{table}

Pour interpréter le sens de ces axes, nous analysons les \textit{loadings} (corrélations entre les variables initiales et les variables canoniques) présentés ci-dessous :

\begin{table}[h!]
    \centering
    \begin{tabular}{lcc|lcc}
        \hline
        \multicolumn{3}{c|}{\textbf{Groupe X (Stratégie)}} & \multicolumn{3}{c}{\textbf{Groupe Y (Performance)}} \\
        \textbf{Variable} & \textbf{CV1} & \textbf{CV2} & \textbf{Variable} & \textbf{CV1} & \textbf{CV2} \\
        \hline
        Disponibilité (365) & \textbf{0.80} & 0.55 & Nb Reviews & \textbf{0.97} & 0.00 \\
        Listings Count & -0.24 & \textbf{0.94} & Prix (log) & 0.00 & \textbf{0.99} \\
        Minimum Nights & -0.29 & 0.33 & Reviews/Mois & \textbf{0.90} & -0.13 \\
        \hline
    \end{tabular}
    \caption{Structure des Loadings : Définition des Axes Économiques}
    \label{tab:cca_loadings}
\end{table}

\section{Interprétation Économique}

Les résultats de la CCA valident l'existence d'une fracture du marché en deux modèles professionnels distincts :

\begin{enumerate}
    \item \textbf{L'Axe 1 : La Para-hôtellerie de Volume (CV1)} \\
    Cet axe montre une corrélation forte entre la \textbf{haute disponibilité} (0.80) et le \textbf{volume d'avis} (0.97). Le prix ne joue aucun rôle sur cet axe (~0.00). Cela isole le comportement para-hôtelier : une stratégie de flux tendu et de rotation rapide, caractéristique des "usines à touristes".

    \item \textbf{L'Axe 2 : La Rente Immobilière d'Investissement (CV2)} \\
    Cet axe, indépendant du premier, lie exclusivement la \textbf{multi-propriété} (0.94) au \textbf{prix élevé} (0.99). Les avis sont absents de cette équation. Cela isole le comportement de l'investisseur : une stratégie de rente basée sur la valorisation d'un parc immobilier et la marge, plutôt que sur l'intensité d'usage.
\end{enumerate}

\textbf{Conclusion :} L'analyse démontre que le marché Airbnb à New York n'est pas homogène. Il est structuré par deux forces de professionnalisation (le Volume et la Rente) qui s'éloignent du modèle originel de l'économie de partage.

\newpage

% ====================================================================
% PARTIE 4 : CLUSTERING (SEGMENTATION)
% Étudiant responsable : [Nom Étudiant 4]
% ====================================================================
\chapter{Segmentation des Hôtes (Clustering)}
\textit{Responsable : Étudiant 4}

\section{Classification K-Means}
Sur la base des composantes principales, nous avons identifié 4 clusters distincts d'annonces.

\section{Profils Types}
\instruction{Étudiant 4}{Décrire les clusters (voir slide 15/16).}
\begin{itemize}
    \item \textbf{Cluster 0 :} Offre standard moyenne gamme.
    \item \textbf{Cluster 1 (Business) :} Disponibilité maximale, géré par des multi-propriétaires.
    \item \textbf{Cluster 2 (Partage) :} Faible disponibilité, prix bas, hôtes occasionnels.
    \item \textbf{Cluster 3 :} Offre luxe / long séjour.
\end{itemize}

\section{Analyse Croisée}
\todo{Insérer les graphiques "stacked bar" montrant la répartition des types d'appartements et des arrondissements par cluster.}

\newpage

% ====================================================================
% PARTIE 5 : ANALYSE SPATIALE ET MODÉLISATION
% Étudiant responsable : [Nom Étudiant 5]
% ====================================================================
\chapter{Analyse Spatiale et Modélisation}
\textit{Responsable : Étudiant 5}

\section{DBSCAN : La Réalité Géographique}
\instruction{Étudiant 5}{Commenter le clustering par densité.}

L'algorithme DBSCAN révèle un "super-cluster" touristique ignorant les frontières administratives entre Manhattan et Brooklyn, concentrant l'activité professionnelle.

\section{Régression des Prix (OLS)}
\instruction{Étudiant 5}{Analyser les déterminants du prix.}

Le modèle économétrique ($R^2 \approx 0.50$) montre que :
\begin{itemize}
    \item La localisation à Manhattan est le facteur n°1 (+55\% sur le prix).
    \item La privatisation du logement (Entire home) est le facteur n°2.
    \item Les professionnels ne sont pas intrinsèquement plus chers à bien égal, mais génèrent plus de volume.
\end{itemize}

\chapter*{Conclusion Générale}
L'étude confirme la dualité du marché new-yorkais entre une périphérie de partage et un centre hyper-professionnalisé.

\end{document}
